%-------------------------------------------------
% FileName: chapt-2.tex
% Author: Safin (zhaoqid@zsc.edu.cn)
% Version: 0.1
% Date: 2020-05-12
% Description: 第2章
% Others: 
% History: origin
%------------------------------------------------- 

% 断页
% \clearpage
\chapter{RISC-V处理器介绍}

\section{RISC-V处理器指令集}

\begin{enumerate}[label={(\arabic*)},itemsep=0pt, parsep=0pt]
	\item 指令集格式

	      RISC-V 基础指令集(Base ISA)作为 RISC-V 架构的核心,为处理器提供了基本的指令集框架。如表\ref{tab:riscv_instruction_formats}和图\ref{fig:riscv_instruction_formats}所示,RV32I 包含了六种基本指令类型:
	      \begin{enumerate}[]
		      \item \textbf{R型}:两个操作数来自寄存器,用于寄存器间操作;
		      \item \textbf{I型}:两个操作数来自立即数和寄存器,用于实现立即数和取数操作;
		      \item \textbf{S型}:两个操作数来自寄存器,一个操作数来自立即数,用于实现访存操作;
		      \item \textbf{B型}:两个操作数来自寄存器,一个操作数来自立即数,用于实现条件分支操作;
		      \item \textbf{U型}:一个操作数来自立即数,用于实现长立即数操作;
		      \item \textbf{J型}:一个操作数来自立即数,用于实现无条件跳转操作。
	      \end{enumerate}

	      从指令格式就可以看出RISC-V指令集最大的优点——精简。首先,指令的长度都为32位,同一种类型的指令格式单一,大幅度减小了译码器的开销以及实现难度。其次,R型指令提供了三个寄存器,这对于需要三个操作数的指令不需要额外的访存,避免了访存带来的开销。然后,寄存器的编码都在指令的固定位置(rs1和rs2),在译码之前就可以先读取寄存器的值读。最后,立即数的符号位被编码至指令的最高位,所以,立即数的符号扩展操作可以与译码操作并行处理。

	      %绘制RV32I指令结构
	      \begin{figure}[htbp]
		      \centering % 使图像居中
		      \begin{adjustbox}{width=0.9\textwidth} % 设置图像宽度为页面宽度的 0.9 倍
			      % 这里是你的 bytefield 表格代码
			      \begin{bytefield}[bitwidth=1.2em, boxformatting={\centering}, endianness=big]{32}
				      % R-type (Register-Register)
				      \bitheader{31,25,24,20,19,15,14,12,11,7,6,0} \\ % 显示边界条件的坐标
				      \bitbox{7}{funct7} & \bitbox{5}{rs2} & \bitbox{5}{rs1} & \bitbox{3}{funct3} & \bitbox{5}{rd} & \bitbox{7}{opcode} \\
				      \wordbox[]{1}{R-type (Register-Register)} \\

				      % I-type (Immediate)
				      \bitheader{31,20,19,15,14,12,11,7,6,0} \\
				      \bitbox{12}{imm[11:0]} & \bitbox{5}{rs1} & \bitbox{3}{funct3} & \bitbox{5}{rd} & \bitbox{7}{opcode} \\
				      \wordbox[]{1}{I-type (Immediate)} \\

				      % S-type (Store)
				      \bitheader{31,25,24,20,19,15,14,12,11,7,6,0} \\
				      \bitbox{7}{imm[11:5]} & \bitbox{5}{rs2} & \bitbox{5}{rs1} & \bitbox{3}{funct3} & \bitbox{5}{imm[4:0]} & \bitbox{7}{opcode} \\
				      \wordbox[]{1}{S-type (Store)} \\

				      % B-type (Branch)
				      \bitheader{31,25,24,20,19,15,14,12,11,7,6,0} \\
				      \bitbox{7}{imm[12|10:5]} & \bitbox{5}{rs2} & \bitbox{5}{rs1} & \bitbox{3}{funct3} & \bitbox{5}{imm[4:1|11]} & \bitbox{7}{opcode} \\
				      \wordbox[]{1}{B-type (Branch)} \\

				      % U-type (Upper Immediate)
				      \bitheader{31,12,11,7,6,0} \\
				      \bitbox{20}{imm[31:12]} & \bitbox{5}{rd} & \bitbox{7}{opcode} \\
				      \wordbox[]{1}{U-type (Upper Immediate)} \\

				      % J-type (Jump)
				      \bitheader{31,12,11,7,6,0} \\
				      \bitbox{20}{imm[20|10:1|11|19:12]} & \bitbox{5}{rd} & \bitbox{7}{opcode} \\
				      \wordbox[]{1}{J-type (Jump)} \\
			      \end{bytefield}
		      \end{adjustbox}
		      \caption{RV32I 指令集格式} % 添加题注
		      \label{fig:riscv_instruction_formats}
	      \end{figure}

	      %绘制指令类型介绍
	      \begin{table}[htbp]
		      \centering
		      \caption{RISC-V 指令格式介绍}
		      \begin{tabularx}{\textwidth}{>{\centering\arraybackslash}X >{\centering\arraybackslash}X >{\centering\arraybackslash}X}
			      \toprule
			      \textbf{类型}   & \textbf{用途} & \textbf{示例指令}              \\
			      \midrule
			      R型(寄存器-寄存器操作) & 算术和逻辑运算     & \texttt{add x1, x2, x3}    \\
			      % \cmidrule{1-3}
			      I型(立即数操作)     & 加载、立即数操作和跳转 & \texttt{addi x1, x2, 10}   \\
			      % \cmidrule{1-3}
			      S型(存储操作)      & 数据从寄存器存储到内存 & \texttt{sw x1, 12(x2)}     \\
			      % \cmidrule{1-3}
			      B型(条件分支)      & 条件分支跳转      & \texttt{beq x1, x2, label} \\
			      % \cmidrule{1-3}
			      U型(高位立即数操作)   & 加载高位立即数     & \texttt{lui x1, 0x12345}   \\
			      % \cmidrule{1-3}
			      J型(无条件跳转)     & 函数调用或长跳转    & \texttt{jal x1, label}     \\
			      \bottomrule
		      \end{tabularx}
		      \label{tab:riscv_instruction_formats}
	      \end{table}

	\item 通用寄存器

	      如图\ref{tab:riscv_instruction_register}表示,RISC-V有32个寄存器,特殊的是,x0寄存器硬连线为0,可替代约15\%的指令操作,而X86需要显式XOR清零。为了提升处理器的性能,数据应该尽量存储在寄存器中,但是频繁的恢复和保存寄存器需要不断地访问内存,会带来不小的开销。为了避免这种情况,RISC-V的处理方案是设置临时寄存器(t0-6)和保存寄存器(s0-11)。临时寄存器的值不需要保存至内存,而保存寄存器的值需要保存至内存。

	      相比X86的16个寄存器,RISC-V架构的寄存器数量多一倍,适当地提升寄存器的数量,处理器可以充分地调度更多的寄存器,以至于加快程序编译和运行的速度。但是,寄存器并不是越多越好,由于其制造工艺和其特殊性导致了寄存器的成本昂贵,所以更多的寄存器会导致更高的成本和硬件复杂度,指令集也需要更多的比特位来对寄存器进行编码,一定程度上压缩了其余字段的编码空间,会提升编码和译码的复杂度。

	      %绘制寄存器表格
	      \begin{table}[htbp]
		      \centering
		      \caption{RISC-V 寄存器介绍}
		      \begin{tabularx}{\textwidth}{>{\centering\arraybackslash}X >{\centering\arraybackslash}X >{\centering\arraybackslash}X}
			      \toprule
			      \textbf{寄存器} & \textbf{名称} & \textbf{功能} \\
			      \midrule
			      x0           & zero        & 始终为0        \\
			      x1           & ra          & 返回地址        \\
			      x2           & sp          & 栈指针         \\
			      x3           & gp          & 全局指针        \\
			      x4           & tp          & 线程指针        \\
			      x5           & t0          & 临时寄存器/链接寄存器 \\
			      x6-7         & t1-2        & 临时寄存器       \\
			      x8           & fp / s0     & 帧指针/保存寄存器   \\
			      x9           & s1          & 保存寄存器       \\
			      x10-11       & a0-1        & 函数参数/返回值    \\
			      x12-17       & a2-7        & 函数参数        \\
			      x18-27       & s2-11       & 保存寄存器       \\
			      x28-31       & t3-6        & 临时寄存器       \\
			      \bottomrule
		      \end{tabularx}
		      \label{tab:riscv_instruction_register}
	      \end{table}

	\item 指令集扩展

	      RISC-V 指令集的另一个显著特点是模块化设计。RV32I 作为基础 ISA,虽然只定义了47条指令,但是足以支持运行基本的软件,其稳定性为开发者提供了可靠的指令集基础。模块化设计允许开发者根据具体需求选择特定的扩展,例如:
	      \begin{enumerate}[]
		      \item \textbf{RV32M}:支持乘除法扩展,从基本指令集分离出来的一个单独标准,需要设计对应的乘除法单元,适用于嵌入式系统、低功耗微控制器、高效处理整数运算的场景。
		      \item \textbf{RV32A}:支持原子指令扩展,对共享内存的数据进行操作的一种方式,能够保证多线程并发执行的一致性,适用于多核处理器、操作系统内核、实时系统的场景。
		      \item \textbf{RV32F}:支持单精度浮点扩展,新增了浮点寄存器,支持单精度浮点的传输、比较、转换、分类,适用于图形处理、传感器数据处理、轻量级机器学习推理的场景。
		      \item \textbf{RV32D}:支持双精度浮点扩展,将浮点寄存器扩展至64位,新增了支持双精度浮点数相关的运算,适用于科学计算、高精度工程仿真、复杂模型训练的场景。
	      \end{enumerate}

	      这种灵活的扩展机制使得 RISC-V 能够适应从嵌入式系统到高性能计算的多样化应用场景,如:物联网设备只需 RV32M + RV32F,而科学计算需要 RV32F + RV32D。这种设计理念为芯片设计者、软件开发者和终端用户带来了多方面的好处。

\end{enumerate}
\section{RISC-V处理器相关技术}

