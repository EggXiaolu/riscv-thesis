%-------------------------------------------------
% FileName: abstract-ch.tex
% Author: Safin (zhaoqid@zsc.edu.cn)
% Version: 0.1
% Date: 2020-05-12
% Description: 中文摘要
% Others: 
% History: origin
%------------------------------------------------- 
% 以下不用改动-------------------------------------
% 断页
\clearpage
% 页码从1开始计数
\setcounter{page}{1}
% 大写罗马数字显示页码
\pagenumbering{Roman}
% 加入书签, bm@abstractname要唯一
\currentpdfbookmark{\defabstractname}{bm@abstractname}
% \chapter*{} 表示不编号,不生成目录
% \markboth{}{} 用于页眉
% 此处以中文题目作为章题目 2024版废止
% \chapter*{\deftitle\markboth{\defabstractname}{}}
% 以摘要为章题目
\chapter*{\defabstractname\markboth{\defabstractname}{}}



% 修改摘要和关键词---------------------------------
% 中文摘要
\abstract{
	本文提出了一种基于 RISC-V 架构的处理器仿真设计方法,专注于 RV32E 指令集的高效实现。研究从处理器核心的内部结构出发,系统地介绍了取指单元、译码单元、计算单元、访存单元和写回单元的设计细节。通过合理连接这些关键单元,构建了一个简单的单周期处理器架构。在此基础上,引入握手信号以优化时序逻辑,进一步将单周期处理器升级为多周期处理器,显著提升了处理器的时序灵活性与性能表现。

	进一步地,本文采用 ARM 的 AXI4 总线协议,实现了处理器与外部设备的高效连接,并成功将其集成到片上系统(SoC)中。为了进一步优化处理器性能,处理器引入了流水线技术和高速缓存技术,显著提升了处理器的并行处理能力和内存访问速度,使其能够更好地满足现代计算任务的需求。

	在验证环节,本文详细探讨了处理器的模拟仿真过程。通过使用开源工具 Verilator 将 RTL 代码编译为 C++ 程序并进行仿真运行,确保了仿真的高效性和灵活性。同时,研究使用一种高效的差分测试方法(Differential testing),通过与行为模拟器的状态比对,实现了测试的高效性和准确性,从而确保了设计的高效性和可靠性。

	综上所述,本文的研究为 RISC-V 架构处理器的设计与验证提供了一种系统化、高效化的方法,具有重要的理论意义和实践价值,为后续的处理器设计与优化工作奠定了坚实的基础。
}

% 中文关键词
% 关键词是供检索用的主题词条,应采用能覆盖毕业设计(论文)主要内容的通用技术词条(参照相应的技术术语标准)。关键词一般为3~5个,每个关键词不超过5个字。
\keywords{RISC-V;体系结构;性能优化;设计方法}


