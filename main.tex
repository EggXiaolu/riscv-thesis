%-------------------------------------------------
% FileName: main.tex
% Author: Safin (zhaoqid@zsc.edu.cn)
% Version: 0.1
% Date: 2020-05-12
% Description: 主文档,包含其它tex文档
%     如果要增加新的章节,请先到 tex目录增加文件,再input
%     如果附录没有内容,就注释掉
% Others: 编译方法参考 make.sh
%         适配TeXLive2020
%         必须用 [UTF-8无BOM] 编码
% History: origin
%-------------------------------------------------

\documentclass{style/zscthesis} % 自定义thesis

% 在正式撰写论文之前,请移步
% https://gitee.com/yeyunxiaopan/zsc-cs-latex-thesis
% 把模板的使用教程简单浏览一下
% 也可以对照B站视频一起看
% https://www.bilibili.com/video/av328559652

% Latex模板撰写的论文最终输出成PDF文档,当然不可能和Word一模一样。
% 如果觉得表格编辑有点麻烦,可以使用[在线工具](https://www.tablesgenerator.com/)。
% 如果图片导致文中有大段空白,可以通过调整文字和图片的位置,以及设置正确的浮动方式,来抑制文中的大段空白。
% 切记,参考文献,一定要在文中引用,才能出现在最后的参考文献章节。
% 切记,图片和表格一定要在文中引用,千万不能写,如下图,如下表。

% 根据教务处2024年规定,封面由维普系统生成,可以在论文中将其注释掉

\begin{document}
%-------------------------------------------------
% FileName: frontinfo.tex
% Author: Safin (zhaoqid@zsc.edu.cn)
% Version: 0.1
% Date: 2020-05-12
% Description: 封面
% Others: 
% History: origin
%------------------------------------------------- 

% 根据个人信息修改{}中的内容------------------------
% 论文题目
\mytitle{基于RISC-V的处理器仿真设计}
% 教学指导单位 
\institute{计算机与人工智能学院}
% 专业名称 
\major{网络工程}
% 班级
\class{21网工1班}
% 学号 多个学号用~分隔,如果要分行用\\分隔
\studentid{21211835111}
% 学生姓名 多个学生用~分隔,如果要分行用\\分隔
\student{卢宇鑫}
% 指导教师(职称) 多个导师用~分隔,如果要分行用\\分隔
\advisor{金可仲}
% 指导单位 
% \institute{计算机学院}
% 完成时间  
\completedate{\today}

% 以下不用改动-------------------------------------
% 加入书签, bm@frontpage要唯一
\currentpdfbookmark{\deffrontpage}{bm@frontpage}
% 封面无页眉页脚
\thispagestyle{empty}
% 根据以上的参数 生成标题页
\mymaketitle

       % 封面,如不需要可以注释掉
%-------------------------------------------------
% FileName: abstract-ch.tex
% Author: Safin (zhaoqid@zsc.edu.cn)
% Version: 0.1
% Date: 2020-05-12
% Description: 中文摘要
% Others: 
% History: origin
%------------------------------------------------- 
% 以下不用改动-------------------------------------
% 断页
\clearpage
% 页码从1开始计数
\setcounter{page}{1}
% 大写罗马数字显示页码
\pagenumbering{Roman}
% 加入书签, bm@abstractname要唯一
\currentpdfbookmark{\defabstractname}{bm@abstractname}
% \chapter*{} 表示不编号,不生成目录
% \markboth{}{} 用于页眉
% 此处以中文题目作为章题目 2024版废止
% \chapter*{\deftitle\markboth{\defabstractname}{}}
% 以摘要为章题目
\chapter*{\defabstractname\markboth{\defabstractname}{}}



% 修改摘要和关键词---------------------------------
% 中文摘要
\abstract{
	本文提出了一种基于 RISC-V 架构的处理器仿真设计方法,专注于 RV32E 指令集的高效实现。研究从处理器核心的内部结构出发,系统地介绍了取指单元、译码单元、计算单元、访存单元和写回单元的设计细节。通过合理连接这些关键单元,构建了一个简单的单周期处理器架构。在此基础上,引入握手信号以优化时序逻辑,进一步将单周期处理器升级为多周期处理器,显著提升了处理器的时序灵活性与性能表现。

	进一步地,本文采用 ARM 的 AXI4 总线协议,实现了处理器与外部设备的高效连接,并成功将其集成到片上系统(SoC)中。为了进一步优化处理器性能,处理器引入了流水线技术和高速缓存技术,显著提升了处理器的并行处理能力和内存访问速度,使其能够更好地满足现代计算任务的需求。

	在验证环节,本文详细探讨了处理器的模拟仿真过程。通过使用开源工具 Verilator 将 RTL 代码编译为 C++ 程序并进行仿真运行,确保了仿真的高效性和灵活性。同时,研究使用一种高效的差分测试方法(Differential testing),通过与行为模拟器的状态比对,实现了测试的高效性和准确性,从而确保了设计的高效性和可靠性。

	综上所述,本文的研究为 RISC-V 架构处理器的设计与验证提供了一种系统化、高效化的方法,具有重要的理论意义和实践价值,为后续的处理器设计与优化工作奠定了坚实的基础。
}

% 中文关键词
% 关键词是供检索用的主题词条,应采用能覆盖毕业设计(论文)主要内容的通用技术词条(参照相应的技术术语标准)。关键词一般为3~5个,每个关键词不超过5个字。
\keywords{RISC-V;体系结构;性能优化;设计方法}


     % 中文摘要
%-------------------------------------------------
% FileName: abstract-en.tex
% Author: Safin (zhaoqid@zsc.edu.cn)
% Version: 0.1
% Date: 2020-05-12
% Description: 英文摘要
% Others: 
% History: origin
%------------------------------------------------- 

% 以下不用改动-------------------------------------
% 断页
\clearpage
% 加入书签, bm@ABSTRACTNAME要唯一
\currentpdfbookmark{\defABSTRACTNAME}{bm@ABSTRACTNAME}
% \chapter*{} 表示不编号,不生成目录
% \markboth{}{} 用于页眉
% 此处以英文题目作为章题目 2024版废止
% \chapter*{\defTITLE \markboth{\defABSTRACTNAME}{}} 
% 以Abstract为章题目
\chapter*{\defABSTRACTNAME\markboth{\defABSTRACTNAME}{}}

% 修改摘要和关键词---------------------------------
% 英文摘要
% 英文摘要与中文摘要的内容应一致。
\ABSTRACT{
	This paper presents a processor simulation design method based on the RISC-V architecture, focusing on the efficient implementation of the RV32E instruction set. Starting from the internal structure of the processor core, the study systematically introduces the design details of the fetch unit, decode unit, execution unit, memory access unit, and write-back unit. By properly connecting these key units, a simple single-cycle processor architecture is constructed. On this basis, handshake signals are introduced to optimize timing logic, further upgrading the single-cycle processor to a multi-cycle processor, which significantly enhances the processor's timing flexibility and performance.

	Furthermore, this paper employs ARM's AXI4 bus protocol to achieve efficient connections between the processor and external devices, successfully integrating it into a System on Chip (SoC). To further optimize the processor's performance, pipeline and high-speed cache technologies are introduced, which significantly improve the processor's parallel processing capabilities and memory access speed, enabling it to better meet the needs of modern computing tasks.

	In the verification section, this paper thoroughly explores the simulation process of the processor. By using the open-source tool Verilator to compile RTL code into C++ programs and run simulations, the efficiency and flexibility of the simulation are ensured. Additionally, an efficient differential testing method is employed, comparing the states with those of a behavioral simulator to achieve high efficiency and accuracy in testing, thereby ensuring the efficiency and reliability of the design.
	In summary, this study provides a systematic and efficient method for the design and verification of RISC-V architecture processors, which holds significant theoretical and practical value and lays a solid foundation for subsequent processor design and optimization work.
}

% 英文关键词
% 每一个英文关键词都必须与中文关键词一一对应。
\KEYWORDS{RISC-V;  Architecture; Performance Optimization; Design Approach}






     % 英文摘要
%-------------------------------------------------
% FileName: content.tex
% Author: Safin (zhaoqid@zsc.edu.cn)
% Version: 0.1
% Date: 2020-05-12
% Description: 目录
% Others: 如无需要,不用修改本文件
% History: origin
%-------------------------------------------------

% 断页
\clearpage
% 加入书签, bm@contentsname要唯一
\currentpdfbookmark{\contentsname}{bm@contentsname}
% 生成目录
\tableofcontents

% 断页
\clearpage
% 加入书签, bm@listfigurename要唯一
\currentpdfbookmark{\listfigurename}{bm@listfigurename}
% 生成图目录
%\listoffigures
{
	% 很弱智的,为每个图目录前加上 "图"
	\let\oldnumberline\numberline
	\renewcommand{\numberline}{\figurename~\oldnumberline}
	\listoffigures
}


% 断页
\clearpage
% 加入书签, bm@listtablename要唯一
\currentpdfbookmark{\listtablename}{bm@listtablename}
% 生成表目录
% \listoftables
{
	% 很弱智的,为每个表目录前加上 "表"
	\let\oldnumberline\numberline
	\renewcommand{\numberline}{\tablename~\oldnumberline}
	\listoftables
}


         % 目录 (如无需要,不用修改)
%-------------------------------------------------
% FileName: chapt-1.tex
% Author: Safin (zhaoqid@zsc.edu.cn)
% Version: 0.1
% Date: 2020-05-12
% Description: 第1章
% Others: 
% History: origin
%------------------------------------------------- 

% 断页
\clearpage
% 页码从1开始计数
\setcounter{page}{1}
% 阿拉伯数字显示页码
\pagenumbering{arabic}

\chapter{绪论}

\section{研究背景}
芯片被誉为现代工业的掌上明珠,是信息时代的基石。自1959年世界上第一颗芯片诞生以来,芯片技术以惊人的速度发展,推动了从个人计算机到智能手机、从数据中心到物联网设备的全面革新。半导体行业也因此成为当今世界最具战略意义和经济价值的行业之一。经过半个多世纪的发展,全球处理器架构市场逐渐形成了两大主流阵营:面向高性能计算的X86架构和面向嵌入式系统的ARM架构。然而,随着技术的不断演进和应用场景的多样化,现有的体系结构逐渐暴露出诸多问题:

\begin{enumerate}[label={\arabic*)},itemsep=0pt, parsep=0pt]
	\item \textbf{复杂指令集架构(CISC)的效率问题}:以X86为代表的复杂指令集架构虽然功能强大,但其指令集冗长且复杂,导致指令执行效率较低。为了实现复杂的指令功能,处理器需要集成更多的晶体管和电路,这不仅增加了芯片的设计难度和制造成本,还显著提高了功耗和发热量,限制了其在低功耗场景中的应用。
	\item \textbf{闭源与授权限制}:X86和ARM架构均属于闭源架构,其核心技术和指令集受到严格的版权保护。使用这些架构需要获得相应公司的授权许可,这不仅增加了研发成本,还限制了中小企业和新兴市场的准入,阻碍了技术的普及和创新。
	\item \textbf{市场垄断与供应商锁定}:由于X86和ARM架构在各自领域的主导地位,相关技术的使用高度依赖于特定供应商的支持。这种市场垄断和供应商锁定的局面导致技术更新缓慢,用户选择受限,进一步抑制了行业的发展活力。
\end{enumerate}

为了解决现有体系结构存在的问题,并顺应现代计算机体系结构设计的发展趋势,RISC-V(开源精简指令集架构)应运而生。RISC-V以其开源、免费、开放和自由的特性,迅速成为全球学术界和工业界关注的焦点。任何个人或组织都可以自由使用、修改和分发RISC-V的设计,这为处理器架构的创新和普及提供了前所未有的机会。

RISC-V的起源可以追溯到20世纪80年代初,当时加州大学伯克利分校的David Patterson教授和斯坦福大学的John Hennessy教授分别提出了精简指令集计算(RISC)理念\cite{cocke1990evolution}。RISC的核心思想是通过简化指令集,使处理器设计更加高效、易于实现和优化。基于这一理念,伯克利分校开发了RISC-I和RISC-II原型机,成功验证了RISC架构的可行性和优越性。2010年,加州大学伯克利分校的Krste Asanović教授及其团队启动了RISC-V项目,旨在设计一种全新的、开放的指令集架构,以满足现代计算的多样化需求\cite{waterman2011risc}。2014年,伯克利团队发布了RISC-V的初始规范,包括32位和64位的基本指令集\cite{asanovic2014instruction},这一规范的发布标志着RISC-V正式进入公众视野。

2015年,RISC-V基金会(RISC-V Foundation)正式成立,吸引了包括谷歌、英特尔、英伟达等全球顶尖科技公司以及众多学术机构和研究组织的加入。2017年,首批基于RISC-V的商用芯片发布,展现了RISC-V在实际应用中的巨大潜力。2020年,为了避免潜在的政治和法律风险,RISC-V基金会迁至瑞士,并更名为RISC-V International,进一步提升了RISC-V的国际化水平和开放性。

如今,RISC-V已成为全球范围内最具活力的开源指令集架构之一,其应用领域涵盖嵌入式系统、物联网、高性能计算、人工智能、数据中心等多个领域。RISC-V的崛起不仅为处理器架构的设计和实现提供了新的思路,也为全球半导体行业注入了新的活力,推动了技术的民主化和创新生态的繁荣发展。


\section{国内外研究现状}
近年来,RISC-V技术以其开源、模块化和可扩展的特性,在全球范围内引发了广泛关注,并取得了令人瞩目的研究成果。从高性能计算到嵌入式系统,从学术研究到商业应用,RISC-V正在迅速崛起,成为处理器架构领域的重要力量。

2019年,西部数据公司(Western Digital)推出了SweRV核心,这是一款高性能的RISC-V处理器核心,专为数据中心和存储应用设计,SweRV的发布不仅展示了RISC-V在高性能计算领域的潜力,还标志着RISC-V从学术研究向工业应用的重大跨越\cite{marena2019risc}。2021年,阿里巴巴旗下的平头哥半导体发布了玄铁907处理器,这是一款基于RISC-V架构的高性能处理器,已成功授权给多家企业使用,进一步推动了RISC-V的商业化进程\cite{JCDI202106005}。2024年,中国科学院计算技术研究所推出了``香山''昆明湖架构V2,这是一款开源的高性能RISC-V处理器,其卓越的性能和创新的设计再次证明了RISC-V在技术创新上的巨大潜力\cite{JFYZ202303014}。

在系统级集成方面,Vedran Dakić等人提出了一种异构RISC-V SoC设计,该设计集成了高性能的乱序核心、高能效的顺序核心以及专用加速器,充分展现了RISC-V在灵活性和可扩展性方面的优势\cite{electronics13173494}。此外,Koch等人开发了针对RISC-V的FPGA框架FABulous,为RISC-V处理器的快速原型设计和验证提供了高效的工具支持\cite{10.1145/3431920}。在设计方法学方面,钟等人提出了一种软硬件联合验证的设计方法,通过结合Verilator软件仿真和FPGA硬件验证,显著提高了RISC-V处理器的开发效率和可靠性\cite{SDDZ202411008}。在总线设计方面,郝等人采用ARM公司提出的AHB总线协议,成功实现了系统总线的设计,不仅保持了处理器的高性能,还显著减小了芯片的流片面积\cite{JSGG202020007}。针对浮点数运算的挑战,潘等人提出了一种优化的浮点运算单元(FPU)设计,通过改进算法和硬件结构,显著提升了浮点数运算的效率和精度\cite{JSGG202103009}。

RISC-V的快速发展得益于其独特的技术优势。其开源特性降低了研发成本,为中小企业和研究机构提供了平等的创新机会。模块化设计使得RISC-V能够灵活适应不同的应用场景,从低功耗嵌入式设备到高性能计算服务器均可覆盖。RISC-V的可扩展性为定制化设计提供了广阔的空间,特别是在人工智能、物联网和边缘计算等新兴领域。

\section{技术对比}
在现代处理器架构中,RISC(精简指令集计算)和CISC(复杂指令集计算)是两种主要的设计理念。以RISC-V、X86和ARM为例,RISC-V作为RISC架构的代表,展现出显著的优势,尤其是在开源性、模块化设计、可扩展性、功耗和性能等方面。

在开源性方面,RISC-V的完全开源特性使其在灵活性和可扩展性方面远超X86和ARM。这种开源性不仅降低了开发成本,还允许开发者根据具体需求进行定制和优化,从而推动了创新和多样化应用的开发。相比之下,X86和ARM均为非开源架构,需要授权才能使用,这在一定程度上限制了其在特定领域的应用。

在处理器设计方面,RISC-V的模块化设计和高可扩展性是其显著优势。它允许开发者根据不同的应用场景灵活选择和扩展指令集,从而更好地适应从嵌入式系统到高性能计算的多样化需求。而X86架构由于其复杂指令集和非模块化设计,在可扩展性方面表现较差,难以满足新兴应用的快速变化需求。

在功耗方面,RISC-V的设计理念使其在低功耗场景中表现出色。其简洁的指令集和高效的执行效率使得处理器能够在较低的功耗下运行,这对于移动设备和物联网应用尤为重要。相比之下,X86架构由于其复杂的设计和较高的功耗,在移动和嵌入式领域存在明显劣势,尽管其在高性能计算中仍具有一定的优势。

在性能方面,RISC-V通过高效的指令执行和流水线设计,能够实现高性能处理。虽然X86和ARM在某些应用场景中也能提供高性能,但RISC-V的高效设计使其在功耗和性能的平衡上更具优势。此外,RISC-V的开源性和模块化设计进一步增强了其在性能优化方面的潜力。

综上所述,RISC-V作为RISC架构的代表,在开源性、模块化设计、可扩展性、功耗和性能等方面展现出显著的优势,使其在现代处理器架构中具有广阔的应用前景。

\begin{table}[htbp]
	\centering
	\caption{指令集对比}
	\begin{tabularx}{\textwidth}{>{\centering\arraybackslash}X >{\centering\arraybackslash}X >{\centering\arraybackslash}X >{\centering\arraybackslash}X}
		\toprule
		\textbf{特性} & \textbf{RISC-V} & \textbf{X86} & \textbf{ARM} \\
		\midrule
		类型          & RISC            & CISC         & RISC         \\
		开源性         & 开源              & 闭源           & 闭源           \\
		模块化         & 支持              & 不支持          & 不支持          \\
		扩展性         & 高               & 低            & 低            \\
		功耗          & 低               & 高            & 低            \\
		性能          & 高               & 高            & 高            \\
		\bottomrule
	\end{tabularx}
	\label{tab:instruction-set-comparison}
\end{table}

\section{本章小结}
本章从研究背景、国内外研究现状以及技术对比三个方面对RISC-V架构进行了全面介绍。首先,通过对现有处理器架构(如X86和ARM)的分析,指出了复杂指令集架构(CISC)的效率问题、闭源与授权限制以及市场垄断与供应商锁定等问题。这些问题的存在促使了开源精简指令集架构(RISC-V)的诞生。RISC-V以其开源、免费、开放和自由的特性,迅速成为全球学术界和工业界关注的焦点,为处理器架构的创新和普及提供了前所未有的机会。

在国内外研究现状方面,近年来RISC-V技术取得了显著的进展。从嵌入式系统到高性能计算,从学术研究到商业应用,RISC-V正在迅速崛起,成为处理器架构领域的重要力量。众多企业和研究机构纷纷推出了基于RISC-V的处理器和相关技术,进一步推动了RISC-V的商业化进程和技术创新。

在技术对比部分,通过对RISC-V、X86和ARM三种架构的详细对比,展示了RISC-V在开源性、模块化设计、可扩展性、功耗和性能等方面的优势。RISC-V的完全开源特性降低了开发成本,其模块化设计和高可扩展性使其能够灵活适应多样化的应用场景。在功耗方面,RISC-V的设计理念使其在低功耗场景中表现出色,而其高效的指令执行和流水线设计则使其在性能上具有显著优势。

总的来说,RISC-V具有极高的发展潜力,且逐渐成为X86和ARM双足鼎立之后的``第三极''。          % 第1章
%-------------------------------------------------
% FileName: chapt-2.tex
% Author: Safin (zhaoqid@zsc.edu.cn)
% Version: 0.1
% Date: 2020-05-12
% Description: 第2章
% Others: 
% History: origin
%------------------------------------------------- 

% 断页
% \clearpage
\chapter{RISC-V处理器介绍}

\section{RISC-V处理器指令集}

RISC-V 基础指令集(Base ISA)作为 RISC-V 架构的核心,为处理器提供了基本的指令集框架。如图\ref{fig:riscv_instruction_formats}所示,RV32I 包含六种基本指令类型,分别是 R 型、I 型、S 型、B 型、U 型和 J 型。

与 X86 架构的指令集相比,RISC-V 的指令集格式更为简洁且高效。RISC-V 仅定义了六种指令格式,且每条指令长度固定为 32 位或 64 位,这极大地降低了指令译码时的复杂度和开销。相比之下,X86 架构由于其CISC的特性,指令长度不定,每次取指需按照最大指令字长读取,并在译码阶段进行分割,这无疑增加了指令处理的复杂度。此外,RV32I 的基础指令数量仅为 47 条,即使加上乘除法扩展(RV32M),指令总数也不超过 60 条。而 X86 架构在发布初期有 80 条指令,到 2015 年,其指令数量已增长至 1338 条,增加了 16 倍。

RISC-V 指令集的另一个显著特点是其模块化设计。RV32I 作为基础 ISA,足以支持基本的软件运行,其稳定性为开发者提供了可靠的指令集基础。模块化设计允许开发者根据具体需求选择特定的扩展,例如:RV32M 支持乘除法运算,RV32F 支持单精度浮点数运算,RV32D 支持双精度浮点数运算,RV32A 支持原子操作等。这种灵活的扩展机制使得 RISC-V 能够适应从嵌入式系统到高性能计算的多样化应用场景。

%绘制RV32I指令结构
\begin{figure}[htbp]
	\centering % 使图像居中
	\begin{adjustbox}{width=0.9\textwidth} % 设置图像宽度为页面宽度的 0.9 倍
		% 这里是你的 bytefield 表格代码
		\begin{bytefield}[bitwidth=1.2em, boxformatting={\centering}, endianness=big]{32}
			% R-type (Register-Register)
			\bitheader{31,25,24,20,19,15,14,12,11,7,6,0} \\ % 显示边界条件的坐标
			\bitbox{7}{funct7} & \bitbox{5}{rs2} & \bitbox{5}{rs1} & \bitbox{3}{funct3} & \bitbox{5}{rd} & \bitbox{7}{opcode} \\
			\wordbox[]{1}{R-type (Register-Register)} \\

			% I-type (Immediate)
			\bitheader{31,20,19,15,14,12,11,7,6,0} \\
			\bitbox{12}{imm[11:0]} & \bitbox{5}{rs1} & \bitbox{3}{funct3} & \bitbox{5}{rd} & \bitbox{7}{opcode} \\
			\wordbox[]{1}{I-type (Immediate)} \\

			% S-type (Store)
			\bitheader{31,25,24,20,19,15,14,12,11,7,6,0} \\
			\bitbox{7}{imm[11:5]} & \bitbox{5}{rs2} & \bitbox{5}{rs1} & \bitbox{3}{funct3} & \bitbox{5}{imm[4:0]} & \bitbox{7}{opcode} \\
			\wordbox[]{1}{S-type (Store)} \\

			% B-type (Branch)
			\bitheader{31,25,24,20,19,15,14,12,11,7,6,0} \\
			\bitbox{7}{imm[12|10:5]} & \bitbox{5}{rs2} & \bitbox{5}{rs1} & \bitbox{3}{funct3} & \bitbox{5}{imm[4:1|11]} & \bitbox{7}{opcode} \\
			\wordbox[]{1}{B-type (Branch)} \\

			% U-type (Upper Immediate)
			\bitheader{31,12,11,7,6,0} \\
			\bitbox{20}{imm[31:12]} & \bitbox{5}{rd} & \bitbox{7}{opcode} \\
			\wordbox[]{1}{U-type (Upper Immediate)} \\

			% J-type (Jump)
			\bitheader{31,12,11,7,6,0} \\
			\bitbox{20}{imm[20|10:1|11|19:12]} & \bitbox{5}{rd} & \bitbox{7}{opcode} \\
			\wordbox[]{1}{J-type (Jump)} \\
		\end{bytefield}
	\end{adjustbox}
	\caption{RV32I 指令集格式} % 添加题注
	\label{fig:riscv_instruction_formats} % 添加标签以便引用
\end{figure}

表\ref{tab:riscv_instruction_formats}介绍了RV32I指令类型的功能和示例:R型用于寄存器间操作、I型用于立即数和取数操作、S型用于存数操作、B型用于条件分支跳转操作、U型用于高位立即数操作、J型用于无条件跳转。

%绘制指令类型介绍
\begin{table}[htbp]
	\centering
	\caption{RISC-V 指令格式介绍}
	\begin{tabularx}{\textwidth}{>{\centering\arraybackslash}X >{\centering\arraybackslash}X >{\centering\arraybackslash}X}
		\toprule
		\textbf{类型}   & \textbf{用途} & \textbf{示例指令}              \\
		\midrule
		R型(寄存器-寄存器操作) & 算术和逻辑运算     & \texttt{add x1, x2, x3}    \\
		% \cmidrule{1-3}
		I型(立即数操作)     & 加载、立即数操作和跳转 & \texttt{addi x1, x2, 10}   \\
		% \cmidrule{1-3}
		S型(存储操作)      & 数据从寄存器存储到内存 & \texttt{sw x1, 12(x2)}     \\
		% \cmidrule{1-3}
		B型(条件分支)      & 条件分支跳转      & \texttt{beq x1, x2, label} \\
		% \cmidrule{1-3}
		U型(高位立即数操作)   & 加载高位立即数     & \texttt{lui x1, 0x12345}   \\
		% \cmidrule{1-3}
		J型(无条件跳转)     & 函数调用或长跳转    & \texttt{jal x1, label}     \\
		\bottomrule
	\end{tabularx}
	\label{tab:riscv_instruction_formats}
\end{table}


\section{RISC-V处理器相关技术}
%-------------------------------------------------
% FileName: chapt-3.tex
% Author: Safin (zhaoqid@zsc.edu.cn)
% Version: 0.1
% Date: 2020-05-12
% Description: 第3章
% Others: 
% History: origin
%------------------------------------------------- 

% 断页
% \clearpage 

\chapter{RISC-V处理器设计}

\section{取指单元(IFU)}

\section{译码单元(IDU)}

\section{计算单元(EXU)}

\section{访存单元(LSU)}

\section{写回单元(WBU)}

\section{异常处理}





%-------------------------------------------------
% FileName: chapt-4.tex
% Author: Safin (zhaoqid@zsc.edu.cn)
% Version: 0.1
% Date: 2020-05-12
% Description: 第4章
% Others: 
% History: origin
%------------------------------------------------- 


% 断页
% \clearpage
\chapter{处理器相关技术设计}

\section{AXI4总线}

\section{片上系统(SoC)}

\section{系统优化}

\subsection{存储优化——高速缓存}

\subsection{并行优化——流水线}
%-------------------------------------------------
% FileName: chapt-5.tex
% Author: Safin (zhaoqid@zsc.edu.cn)
% Version: 0.1
% Date: 2020-05-12
% Description: 第5章
% Others: 
% History: origin
%------------------------------------------------- 


% 断页
% \clearpage
\chapter{仿真测试}

\section{verilator仿真}

\section{软硬件差分测试}
%-------------------------------------------------
% FileName: chapt-6.tex
% Author: Safin (zhaoqid@zsc.edu.cn)
% Version: 0.1
% Date: 2020-05-12
% Description: 第6章
% Others: 
% History: origin
%------------------------------------------------- 

% 断页
% \clearpage

\chapter{总结和展望 }

\section{本文总结}

\section{未来展望}




% \begin{lstlisting}[language=TeX,frame=single,]
% { 
%   $\int f(x) \mathrm{d}x$
%   $\sum_{0}^{+\infty}$
%   \begin{center}
%   居中
%   \end{center}
% }
% \end{lstlisting}

          % 第6章
%-------------------------------------------------
% FileName: reference.tex
% Author: Safin (zhaoqid@zsc.edu.cn)
% Version: 0.1
% Date: 2020-05-12
% Description: 参考文献
% Others: 如无需要,不用修改本文件
%         参考文献请到 bib/ref.bib中按格式增加
% History: origin
%-------------------------------------------------


% 参考文献是毕业设计(论文)不可缺少的组成部分,在毕业设计(论文)的撰写过程中应承认和尊重他人的知识成果,参考与引用的内容必须注明,杜绝抄袭、剽窃他人成果。同时,引用的资料应具有权威性,并对毕业设计(论文)有直接的参考价值。
% 要求查阅文献15篇(含)以上,其中外文文献3篇(含)以上,近三年公开发表的文献3篇(含)以上,书籍不超过5本,期刊([J])和论文集([C])8篇(含)以上,包括导师指定的全部参考文献。


% 断页
\clearpage

% hyperef 精确定位
% 设置一个anchor,主要针对\addcontentsline
% 防止目录,书签等指向错误位置 
\phantomsection

% 增加到目录,与chapter同级别
\addcontentsline{toc}{chapter}{\bibname}

%------------------------------------------
% 默认参考文献样式 plain
% \bibliographystyle{plain}
% gbt7714 顺序编码制
\bibliographystyle{bib/gbt7714-numerical} 
%------------------------------------------

% ref是BIBTEX数据库的文件名,不要带.bib 扩展名,实际文件 bib/ref.bib
\bibliography{bib/ref}




         % 参考文献 (修改bib/ref.bib)
%-------------------------------------------------
% FileName: acknowledgement.tex
% Author: Safin (zhaoqid@zsc.edu.cn)
% Version: 0.1
% Date: 2020-05-12
% Description: 致谢
% Others: 
% History: origin
%------------------------------------------------- 

% 以下不用改动-------------------------------------
% 断页
\clearpage
% hyperef 精确定位
% 设置一个anchor,主要针对\addcontentsline
% 防止目录,书签等指向错误位置 
\phantomsection
%增加到目录,与chapter同级别
\addcontentsline{toc}{chapter}{\defacknowledgement}
% \chapter*{} 表示不编号,不生成目录
% \markboth{}{} 用于页眉
\chapter*{\defacknowledgement \markboth{\defacknowledgement}{}}

% 修改{}中的内容---------------------------------
\acknowledgement{
	谢谢!
}

 % 致谢
%-------------------------------------------------
% FileName: appendix.tex
% Author: Safin (zhaoqid@zsc.edu.cn)
% Version: 0.1
% Date: 2020-05-12
% Description: 附录
% Others: 如果没有内容,就在main.tex中注释掉
% History: origin
%------------------------------------------------- 


% 以下不用改动-------------------------------------
% 断页
% \clearpage



% hyperef 精确定位
% 设置一个anchor,主要针对\addcontentsline
% 防止目录,书签等指向错误位置 
% \phantomsection
% 增加到目录,与chapter同级别
% \addcontentsline{toc}{chapter}{\appendixname} 
% \chapter*{} 表示不编号,不生成目录
% \markboth 用于页眉
% \chapter*{\appendixname \markboth{\appendixname}{}} 

% appendix 用于附录章节的特殊编号 从A开始
\appendix

% % 附录章
% \chapter{算法} 

% % 修改以下内容---------------------------------
% 对于一些不宜放入正文,又是毕业设计(论文)不可缺少的部分,或有重要参考价值的内容,可编入附录中。例如:过长的公式推导,大量的数据和图表,程序全文及其说明等。

% \section{xxx} 

% \section{cccc} 

% \chapter{源码} 



% \chapter{数据} 

        % 附录 (无内容请注释掉) 
\end{document}
